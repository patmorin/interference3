\documentclass{patmorin}
\usepackage[T1]{fontenc}
\usepackage[utf8]{inputenc}
\usepackage{amsmath,amsfonts,amsthm}
\usepackage{graphicx}
\usepackage{natbib}

\usepackage{paralist}

\usepackage{pat}

\title{\MakeUppercase{Low-Interference Networks in 1D (and beyond?)}%
    \thanks{This work was partly funded by NSERC.}}

\author{TBD}


\begin{document}
\maketitle


\begin{abstract}
  
\end{abstract}

\section{Introduction}

A \emph{geometric graph} $G$ is a (simple, undirected) graph whose vertex set $V(G)$ is a subset of $\R^d$ and where each edge $vw$ in the edge set $E(G)$ is treated as an open line segment with endpoints $v$ and $w$. 

For any vertex $v\in V(G)$, we define the \emph{transmission radius} $r_G(v)=\max\{ \|vw\| : vw\in E(G) \}$ and the \emph{transmission disk} $d_G(v)=\{ x\in\R^d : \|vx\| \le r(v) \}$.  Intuitively, the transmission radius of $v$ represents how loudly $v$ must broadcast to be heard by all its neighbours.

The (receiver-centric) \emph{interference} at any point $x\in\R^d$ is
\[  
  I_G(x) = |\{v\in V(G): x\in d_G(v) \}| \enspace .
\]
The (maximum receiver-centric) \emph{interference} of $G$ is
\[
   I(G) = \max\{ v\in V(G): I_G(v) \} \enspace ,
\]
and the the (maximum receiver-centric) \emph{interference} of a point set $P\subset\R^d$ is
\[
  \min\{ I(G) : \mbox{$G$ is connected and $V(G)=P$} \}\enspace .
\]

\citet{devroye.morin:note} show that, for dimensions $d=1,2$, and a i.u.d.\ $n$-point set $P\subset\R^d$
\[
  \E[I(P)] \in \Omega\left((\log n)^{1/4}\right) \cap O\left((\log n)^{1/3}\right)  \enspace .
\]
For constant $d\ge 3$, they obtain nearly the same result except for an additional $\log\log n$ factor in the upper bound.

In the current paper, we obtain the tight upper bound
\[
  \E[I(P)] \in O\left((\log n)^{1/4}\right)
\]
in case $d=1$.


\section{The Proof}

First some preliminary lemmas. 


\subsection{Deterministic Results}

The following construction is well-known, but we include a proof for completeness:

\begin{lem}\lemlabel{worst-case}
  For any $k$-point set $P\subset\R$, there exists a connected graph $G$ with $V(G)=P$ such that, for any point $x\in\R$, $I_G(x) = O(\sqrt{k})$.
\end{lem}

\begin{proof}
  Let $P=\{x_1,\ldots,x_k\}$, labelled so that $x_1<\cdots< x_k$.  Let $r=\floor{\sqrt{k}}$ and $t=\floor{(k-1)/r}$.
  Define $Y=\{y_0,\ldots,y_t\}$ so that, for each $i\in\{0,\ldots,t-1\}$, $y_i=x_{1+ir}$, and $y_t=x_k$.  The graph $G$ contains the path $y_0,\ldots,y_t$.  Additionally, $G$ contains an edge from each point $x_i\in P\setminus Y$ to the point in $y_j\in Y$ that minimizes $|x_i-y_j|$.
  
  What remains is to bound $I_G(x)$ for every point $x\in\R$.  The vertices in $Y$ contribute a total of at most $|Y|=t+1$ to $I_G(x)$.  Furthermore, if a point $x_i\in P\setminus Y$ contributes to $I_G(x)$ then there exists some $j\in\{0,\ldots,t-1\}$ such that $y_j \le x,x_i\le y_{j+1}$.  If $x\not\in Y$, then there are at most $r-1$ such points $x_i$. If $x\in Y$, then there are most $2r-2$ such points.  Thus, for any point $x\in\R^d$, $I_G(x)\le t + 2r -1 = O(\sqrt{k})$. 
\end{proof}

% The following lemma helps characterize situations in which the obvious graph produces high interference.
% 
% \begin{lem}\label{zeno-like}
%   Let $G$ be a path $x_1,\ldots,x_k$ with $0<x_1<\cdots<x_k$. Then
%   \begin{compactenum}
%     \item $I_G(0) = |\{i\in\{1,\ldots,k-1\} : x_{i+1}-x_i \ge x_i\}|$.
%     \item Let $J=\{i_0,\ldots,i_s\}\subseteq \{1,\ldots,k\}$ be the set of indices $i$ such that $x_{i+1}-x_i \ge x_i$ ordered so that $i_1<\cdots<i_s$.  Then, for any $j\in\{0,\ldots,s\}$, $x_{i_j+1}-x_{i_j} \ge x_1 2^j$.
%   \end{compactenum}
% \end{lem}
% 
% \begin{proof}
%   Conclusion~1 follows immediately from definitions. For Conclusion 2, we can proceed by induction on $j$. For $j=0$, we have
%   $x_{i_0+1}-x_{i_0} \ge x_{i_0} \ge x_1 =  x_1 2^0$, as required.
% 
%   For any integer $\ell$, let $J_{<\ell}=\{i\in J: i < \ell\}$.  For $j>1$, we have 
%   \[ x_{i_j+1}-x_{i_j} \ge x_{i_j} 
%   = x_1+\sum_{i = 1}^{i_j-1} (x_{i+1}-x_i)
%   \ge x_1+\sum_{i\in J_{<i_j}}(x_{i+1}-x_i) 
%   \ge x_1+x_1(2^{j}-1)
%   = x_12^j \enspace . \qedhere
%   \]
% \end{proof}

\subsection{Probabilistic Results}

Let $z_1,\ldots,z_p$ be obtained from a sequence of independent $\mathrm{Gamma}(q)$ random variables $Z_1,\ldots,Z_p$ where $z_1=0$ and, for each $i\in\{1,\ldots,p\}$, $z_i=\sum_{j=1}^{i-1}Z_i$.  We now study some properties of the path $P_0=z_1,\ldots,z_p$.

We will make use of the following two inequalities that, for any $t>1$, apply to a $\mathrm{Gamma}(q)$ random variable $Z$ (here, $a>0$ is an unspecified constant that I'm too lazy to look up right now):
\begin{equation}
  \Pr\{Z \ge tq\} \le \exp(-a(t-1)q) \eqlabel{gamma-tail-bound}
\end{equation}
and 
\begin{equation}
  \Pr\{Z < q/t\} \le \exp(-aq\log_2 t) \enspace . \eqlabel{gamma-head-bound}
\end{equation}

Fix some value $0< x \le 2Z_1$.  
We say that an index $i\in\{1,\ldots,p\}$ is \emph{bad for $x$} if $z_i < x$ and $Z_i \ge (x-z_i)/2$.  Note that $i$ is bad for $x$ if and only if $z_i<x$ and $z_i$ contributes to $I_{P_0}(x)$.  Let $B_x=\{b_1,\ldots,b_s\}\subseteq\{1,\ldots,p\}$ be the set of indices that are bad for $x$, ordered so that $b_1<\cdots < b_s$.  Observe that $b_i \le b_{i-1}/2$ for each $i\in\{2,\ldots,s\}$.

Consider the event $\mathcal{E}_{t,k}(x)$ defined as follows:
\begin{compactenum}
  \item $Z_1 \le q/2$;
  \item at least $t$ indices among $\{1,\ldots,k\}$ are bad for $x$; and
  \item $z_k \le x$.
\end{compactenum}

\begin{lem}
  $\Pr\{\mathcal{E}_{t,k}\} \le \exp(-\Omega(k+qt^2))$
\end{lem}

\begin{proof}
  The probability that $Z_1,\ldots,Z_k$ satisfies Condition~1 of $\mathcal{E}_{t,k}(x)$ is $\Pr\{Z_1 \le q/2\}$.
  
  Next, consider Condition~2 and
  fix a set $1\le b_1<\cdots<b_t\le k$ of indices. For each $i\in\{b_1,\ldots,b_t\}$, the probability that $b_i$ is bad for $x$ given that $b_{1}\ldots,b_{i-1}$ are each bad for $x$ is at most $\Pr\{Z_{b_i} \le q/2^i\}$.  
  
  Finally, in order to satisfy Condition~3, it must be that $Z_i \le q/2$ for each $i\in\{2,\ldots,k\}\setminus\{b_1,\ldots,b_t\}$.
  Taking the union bound over all $\binom{k}{t}$ choices of $b_1,\ldots,b_t$, we obtain 
  \begin{align*}
    \Pr\{E_{k,t}(x) &
    \le \binom{k}{t}
          \cdot \Pr\{Z \le q/2\}^{k-t}
          \cdot \prod_{i=1}^t\Pr\{Z_{b_i}\le q/2^i\} \\
    & \le \exp(t\ln k)
          \cdot \Pr\{Z \le q/2\}^{k-t}
          \cdot \prod_{i=1}^t\Pr\{Z_{b_i} \le q/2^i\} \\
    & \le \exp(t\ln k - aq(k-t))
          \cdot \prod_{i=1}^t\Pr\{Z_{b_i} \le q/2^i\} \\
    & \le \exp(t\ln k - aq(k-t))
          \cdot \prod_{i=1}^t\exp(-aqi) \\
    & \le \exp\left(t\ln k - aq(k + \binom{t}{2}\right) \\
    & = \exp\left(-\Omega(q(k+t^2))\right)
  \end{align*}
  where the final simplification can be obtained by considering the two cases $\ln k < t$ or $k \ge e^t$.
\end{proof}

Next, consider the possibly more likely event $\mathcal{E}'_{t,k}$ defined in exactly the same way as $\mathcal{E}_{t,k}$ except without Condition 1.  Using \eqref{gamma-tail-bound} and the fact that $Z_{b_i} \le Z_{b_{i-1}/2}$ we obtain the following corollary:

\begin{lem}
  $\Pr\{\mathcal{E}'_{k,t}\}\le \exp(-\Omega(2^t))+\exp(-\Omega(k+qt^2))$.
\end{lem}

Finally, we define $\mathcal{E}'_{t} = \bigcup_{k=t}^n \mathcal{E'}_{k,t}$ and, from the union bound, we obtain:

\begin{lem}\lemlabel{good-one}
  $\Pr\{\mathcal{E}'_{t}\}\le n\exp(-\Omega(2^t))+\exp(-\Omega(qt^2))$.
\end{lem}







\subsection{The Construction}

From this point on, let $X_0,\ldots,X_n$ be a sequence of independent exponential random variables each with mean 1 and for each $i\in\{1,\ldots,n\}$, let $x_i=\sum_{j=0}^{i-1} X_j$.  Let $P=\{x_1,\ldots,x_n\}$.  We will construct a low interference graph $G$, as follows:
\begin{enumerate}
  \item Let $q=\floor{\sqrt{\log n}}$ and $p=\floor{n/q}$.  $G$ contains a path $P_0=z_0,z_1,\ldots,z_p$ where $z_0=x_1$, $z_p=x_n$, and, for each $i\in\{1,\ldots,p-1\}$, $z_i = x_{1+iq}$.
  \item For each $i\in\{0,\ldots,p-1\}$, $G$ contains a graph $G_i$ obtained by applying \lemref{worst-case} to the point set $\{x\in P: z_i \le x \le z_{i+1}\}$.
\end{enumerate}

We claim that, for any $x\in\R$,  $I_G(x) \in O((\log n)^{1/4})$.  

First observe that
\[
    I_G(x) \le I_{P_0}(x) + \sum_{i=0}^p I_{G_i}(x) 
\]
We begin by bounding $I_{P_0}(x)$.

\begin{lem}
  For any $x\in\R$, $\Pr\{I_{P_0}(x) > c(\log n)^{1/4}\} \le n^{-\Omega(c)}$.
\end{lem}

\begin{proof}
  Observe that, if $I_{P_0}(x) \ge 2t$ for some integer $t$, then we may assume, without loss of generality, that there are indices $1\le b_1<\cdots<b_t$ with $z_{b_t} \le x$ such that each of $x_{b_1},\ldots,x_{b_t}$ contributes to $I_{P_0}(x)$.  But this means that, in the sequence $Z_{b_1},Z_{b_1}+1,\ldots,Z_n$, the event $\mathcal{E}'_{t}(x-x_{b_1})$ has occurred.  By \lemref{good-one}, the probability that this occurs is at most
  \[
     \Pr\{\mathcal{E}_t\} \le \exp(-\Omega(2^t)) + \exp(-\Omega(qt^2)) = \exp(-\Omega(c\log n)) = n^{-\Omega(c)}
  \]
  for $t=c(\log n)^{1/4}$.  We complete the proof by applying the union bound over all $n$ possible choices of $b_1$.
\end{proof}

Hmmmm. One issue here is that our construction actually contains a path through every $(\log n)^{1/4}$ vertex. This will probably generate interference $(\log n)^{3/8}$.

  
  
% 
% 
% 
% \end{proof}
% 
% 
% 
% 
% The only obvious quantity in the preceding formula is $I_{G_i(x)}$
% 
% 










\section*{Acknowledgement}

This research was initiated at the Fourteenth Annual Workshop on Probability and Combinatorics at McGill University's Bellairs Research Institute, March 29--April 5, 2019.  The authors are grateful to the workshop organizers and participants for providing a stimulating research environment.


\bibliographystyle{apalike}
\bibliography{interference3}

\end{document}
